%%%%%%%%%%%%%%%%%%%%%%%%%%%%%%%%%%%%%%
% yale_thesis.tex
% Alexander Cerjan
% 2014/04/07
%
% A bare, sample template for a Yale PhD thesis using yalephd.cls
%%%%%%%%%%%%%%%%%%%%%%%%%%%%%%%%%%%%%%

\documentclass[letterpaper,12pt,draft]{yalephd}
% remove draft option for final printing.
% font size must be between 10pt-12pt.

\usepackage{geometry} % you need this for yalephd.cls to work.
\usepackage{graphicx} % you probably want the rest of these.
\usepackage{dcolumn}
\usepackage{bm}
\usepackage{amsmath}
\usepackage{amsfonts}
\usepackage{amssymb}
\usepackage{appendix}
\usepackage{comment}
\usepackage{cite}

\bibliographystyle{plain}

\begin{document}

% Need to define title before the abstract.
\title{Title goes here}
\author{Your name}
\advisor{Your advisor's name}
\date{Date you'll receive your degree} % usually not \today.

% All the stuff at the front of your thesis.
\frontmatter

\begin{abstract}
Abstract goes here. Limit 750 words.
\end{abstract}


\maketitle
\makecopyright{2015}
\tableofcontents
\listoffigures % remove this if you have no figures.
\listoftables % remove this if you have no tables.

\chapter{Acknowledgements} % this needs to be before \mainmatter.
A lot of people are awesome. Probably your family, friends, 
advisor, and that one super special high school teacher who
believed in you.

% Starts proper arabic numbering of pages and chapters.
\mainmatter

\chapter{Introduction}
Your first chapter is probably an introduction. But who knows. Check out Eq.\ \ref{that_right_triangle_rule}!
Note that after Eq.\ and Fig.\ you want to use `.\textbackslash'  to use a single sized space. Otherwise,
latex will interpret it as the end of a sentence and put additional white space in between `Eq.' and 
`\ref{that_right_triangle_rule}'.

\begin{equation}
a^2 + b^2 = c^2 \label{that_right_triangle_rule}
\end{equation}

\chapter{A new chapter}
Your second chapter probably has novel material in it. hopefully.

% Add additional \chapter{}s as necessary.

% use \cite{} to cite a reference in your bibliography file.
% use \ref{} to reference a \label{} from an equation, figure, or table.

% for sets of equations use align:
%\begin{align}
%\end{align}

% for figures:
%\begin{figure}[ht]
%\centering
%\includegraphics[width=.45\textwidth]{name_of_figure.eps}
%\caption{A caption! \label{a_figure}}
%\end{figure}

% for tables:
%\begin{table}
%\begin{tabular}{c|c|c}
% 1 & 2 & 3 \\
%\hline
%\end{tabular}
%\caption{Another caption! \label{a_table}}
%\end{table}

% Only call appendix once, here.
\appendix

\chapter{Stuff}
If you need an appendix, it will go here.

\begin{align}
a^n + b^n &\ne c^n \\
n &> 2
\end{align}

\chapter{More stuff}
A second appendix. Look at you, you over achiever.

% Any chapters such as End Notes go after this.
\backmatter

\bibliography{name_of_your_bibtex_file}
% for your own sake, use a bibtex file, so all of the numbering of references will be done
% automatically.

\end{document}
